\section{Summary}

This work shows several ways how to estimate an integral numerically. This is 
desired if either the integrand is partially unknown, there is no 
antiderivative or its computation is too expensive. At this point numerical 
algorithms provide an easier and approximated solution of the integral.

In the first part, the most important concepts are presented. The basic idea 
of these approaches is to evaluate the integrand at a finite set of points 
and then to sum these values with addition of a weighting function. If the 
function to be integrated is of a low polynomial degree, an exact solution 
can be obtained by the Gaussian quadrature which only requires n points to 
approximate a polynomial of degree $2n-1$. However, if the function has a 
high degree, oscillates or contains singularities, other concepts are needed. 
This often results in more function evaluations and can become expensive 
again. 

A variant to approximating the integral of a function in one step is to subdivide the interval into smaller intervals and determine the integral as a sum of all sub intervals. Especially on periodic functions, the composite midpoint rule becomes extremely accurate. The values of multidimensional intervals can also be determined by using a one-dimensional method in each dimension. This approach is only suitable for low dimensions since the number of evaluations rises exponentially with increasing degree. A technique which evaluates an integral at a random number of points are Monte-Carlo methods.

The second part gives an overview of the routines implemented in numerical 
libraries. As example libraries the Gnu Scientific Library and the library of 
the Numerical Algorithms Group are considered. In the one-dimensional case 
they both mostly rely on QUADPACK, a Fortran library for numerical 
integration of one-dimensional integrals. QAGS and QAGI are two general 
purpose routine which can be used without further analysis of the integrand 
while the former is for the use over finite intervals and the latter for 
infinite intervals. However, there are routines specialized in integrals of functions 
which have singularities or are strongly oscillating. Using such a 
specialized function will be less expensive in most cases since fewer 
conditions need to be checked.

Although the GSL does not natively include methods for multidimensional integration, 
there exists an extension for that purpose. NAGlib provides routines for 
multidimensional integration over hyper rectangles and simplexes, n-cubes and 
n-spheres innately. Moreover, a routine which uses a Monte-Carlo method is 
included as well. With the returned absolute or relative error assumptions of 
the accuracy can be made.




